% -*- mode: latex; eval: (flyspell-mode 1); ispell-local-dictionary: "american"; TeX-master: t; -*-

\documentclass[12pt]{article}

\usepackage{hyperref,amsthm,amsmath,amsfonts,enumerate,stmaryrd}

\newcommand{\bZ}{\mathbb{Z}}
\newcommand{\bN}{\mathbb{N}} %{\bZ_{\ge 0}}
\newcommand{\bNast}{\bN^*}
\newcommand{\binover}{\mathbin{/}}



\newcommand{\seg}[2]{\left\llbracket#1, #2 \right\rrbracket}
\newcommand{\floor}[1]{\left\lfloor #1  \right\rfloor}
\newcommand{\ceil}[1]{\left\lceil #1  \right\rceil}
\newcommand{\abs}[1]{\left| #1 \right|}

\newtheorem{theorem}{Theorem}
\newtheorem{lemma}{Lemma}


 \theoremstyle{definition}
 \newtheorem{definition}{Definition}
 \newtheorem{exercise}{Exercise}
% \newtheorem{example}{Example}




 \begin{document}



 
   
   \begin{theorem} \label{thm:2m-moins-3n}
     For all $m$, $n \in \bN$,
     $- 10 \le 2^m - 3^n  \le 4$ is equivalent to
     $$(m, n) \in \{ (3, 2), (2, 1), (1, 1), (1, 0)  \} \, .
     $$
   \end{theorem}

   
   \begin{proof}
     The idea is to work modulo $80$.
     Put %$G = 80 \bZ$,  
     \begin{align*}
     %I & = \{ -6, -5, -4, - 3, - 2, - 1, 0, 1, 2, 3 \} \,,  \\
     P_2 & = \left\{ 16, 32, 48, 64 \right\} \,,  \\
       P_3 & = \left\{ 1, 3, 9, 27 \right\} \,,
             \intertext{and}
     Q & = P_2 - P_3 - [- 10, 4] = [-15, - 1] \cup [1, 73] \,.
     \end{align*}      
     For each $n \in \bN$,
     $3^{n + 4}$ is congruent to $3^n$ modulo $80$ because 
     $$3^{n + 4} - 3^n =  (3^4 - 1) 3^n = 80 \cdot  3^n \, .
     $$
     Therefore,
     a simple induction on $n$ shows that for each $n \in \bN$,
     at least one element of $P_3$ is congruent to $3^n$ modulo $80$.
     Similarly, for each $m \in \bN + 4$,
     $2^{m + 4}$ is congruent to $2^m$ modulo $80$ because 
     $$
     2^{m + 4} - 2^{m}  = (2^8 - 2^4)2^{m - 4} = 80 \cdot 3 \cdot 2^{m - 4} \,.  
     $$
     %In addition, we have $2^8 - 80 = 48 \in P_2$.
     Therefore, a simple induction on $m$ shows that for each $m \in \bN + 4$,
    at least one element of $P_2$ is congruent to $2^m$ modulo $80$.
     Hence, for every $m \in \bN + 4$, every $n \in \bN$, and every $r \in [-10, 4]$, 
     at least one element of $Q$ is congruent to $2^m - 3^n - r$ modulo $80$.
     Besides, $Q$ does not contain any multiple of $80$.
     %straightforward computation yield $Q = [- 15, - 1] \cup [1, 73]$.
     Therefore,
     $2^m - 3^n$ is incongruent to $r$ modulo $80$
     for every $m \in \bN + 4$,
     every $n \in \bN$, and
     every $r \in [- 10, 4]$.
     Now, let $m$, $n \in \bN$ be such that $2^m - 3^n \in [-10, 4]$.
     The previous discussion ensures $m < 4$,
     whence  
     $$
     3^n \in 2^m - [- 10, 4] \subseteq \{ 1, 2, 4, 8 \} - [- 10, 4] = [ -3, 18] \, ,
     $$
     and consequently, $n \le 2$.
     It follows that $(m, n)$ lies in the small finite set $[0, 3] \times [0, 2]$,
     and thus the desired result is easily checked by inspection.
   \end{proof}

   Since $2^4 - 3^3 = - 11$ and $2^5 - 3^3 = 5$,
   Theorem~\ref{thm:2m-moins-3n} is optimal.
   Its proof relies on the modulus $80$ which is also optimal; let us explain why.
   Put $p = 480720240$ (modular exponentiation is cheap~!);
   let $q \in [2, 79]$.
   If there exists $(m, n) \in \bN \times \bN$ such that
   $q$ divides $3^n - 2^m - 1$, $2^{m + p} - 2^m$, and $3^{n + p} - 3^n$
   then
   $3^{n + pk} - 2^{m + pk}$ is congruent to $1$ modulo $q$ for every $k \in \bN$,
   which makes $q$ unsuitable for our purpose.
   We claim that such an ordered pair $(m, n)$ exists: 
   if $q \notin \{ 16, 27, 32, 48, 54, 64 \}$ then set $(m, n) = (3, 2)$,
   if $q \in \{ 16, 32, 64 \}$ then set $(m, n) = (6, 0)$, 
   if $q \in \{ 27, 54 \}$ then set $(m, n) = (9, 3)$, and 
   if $q = 48$ then set $(m, n) = (5, 4)$.    

   \begin{theorem} \label{thm:pow-2-square-mod}
     For each $m \in \bNast$,
     there exist $\alpha$, $x \in \bNast$ such that for every  $k \in \bNast$, 
     $2^{k \alpha + 3} + 1$ is congruent to $x^2$  modulo~$m$.
   \end{theorem}



   \begin{proof}
     Let $n$, $v \in \bN$ be such that $m = 2^v n$ and $n$ is odd.
     The Chinese remainder theorem ensures that there exists $x \in \bNast$ such that
     $x$ is congruent to $1$ modulo $2^v$ and to $3$ modulo~$n$.
     Since $2$ is multiplicatively invertible modulo $n$,
     there exists $\phi \in \bNast$ such that $2^\phi$ is congruent to $1$ modulo $n$
     (no knowledge about Euler's totient function is required here).
%     Let $\psi \in \bNast$ be such that $\psi \phi - 3 \ge v$; 
     Put $\alpha = (v + 1) \phi$:
     $\alpha$ is positive,
     $\phi$ divides $\alpha$, and
     $\alpha + 3$ is larger than~$v$.
     For each $k \in \bNast$
     $2^{k \alpha + 3} + 1$ is congruent
     to $1$ modulo $2^v$ and
     to $9$ modulo $n$,
     whence $2^{k \alpha + 3} + 1$ is congruent to $x^2$ modulo~$m$.
   \end{proof}
     
   \begin{theorem}
     For each $m \in \bNast$,
     there exist $\alpha$, $x \in \bNast$ such that for every  $k \in \bNast$, 
     $3^{k \alpha + 1} + 1$ is congruent to $x^2$ modulo~$m$.
   \end{theorem}

   \begin{proof}
     The proof is similar to that of Theorem~\ref{thm:pow-2-square-mod} and left to the reader.
    \end{proof} 

    
  
     \begin{lemma} \label{lem:x2=q+1}
       For every $x \in \bN$, $x^2 - 1$ is a prime power if, and only if, $x \in \{ 2, 3 \}$. 
     \end{lemma}

     \begin{proof}
       Since  $2^2 - 1 = 3^1$ and since $3^2 - 1 = 2^3$, the ``if'' part is clear.
       It remains to prove the ``only if'' part.
       Let $x$, $p \in \bN$ be such that $p$ is prime and $x^2 - 1$ is a power of~$p$.
       Since  $x^2 - 1 = (x - 1)(x + 1)$, 
       both $x - 1$ and $x + 1$ can be written as powers of~$p$.
       Besides, $x - 1$ is less than $x + 1$.
       Therefore, $x - 1$ divides $x + 1$.
       It follows that $x - 1$ divides $(x + 1) - (x - 1) = 2$,
       whence $x \in \{ 2, 3 \}$.
    \end{proof} 
    

   

   
   \begin{lemma} \label{lem:2m+3n-square-mod-12}
     For every $m$, $n \in \bNast$ such that $2^m + 3^n$ is a square modulo $12$,
     both $m$ and $n$ are even.
   \end{lemma}

   \begin{proof}
     Let $m$, $n \in \bNast$ be such that $2^m + 3^n$ is a square modulo $12$.
     In particular, $2^m + 3^n$ is both a square modulo $3$ and a square modulo~$4$.
     Since $2^m + 3^n$ is congruent to ${(- 1)}^m$ modulo $3$ and since $- 1$ is not a square modulo $3$,
     $m$ is even.
     In particular, $m$ is not less than $2$, and thus $2^m + 3^n$ is congruent to ${(- 1)}^n$ modulo~$4$.
     Besides, $- 1$ is not a square modulo~$4$.
     Therefore, $n$ is even.
   \end{proof}

   \begin{theorem}
     For every $x$, $m$, $n \in \bN$,
     $x^2 = 2^m + 3^n$ is equivalent to
     $$
     (x, m, n) \in \{  (5, 4, 2), (2, 0, 1), (3, 3, 0) \} \, .
     $$
   \end{theorem}
   
   \begin{proof}
     Assume $x^2 = 2^m + 3^n$.
     If $m = 0$ then $x^2 - 1 = 3^n$ and thus Lemma~\ref{lem:x2=q+1} ensures $x = 2$ and $n = 1$;
     if $n = 0$ then $x^2 - 1 = 2^m$ and thus Lemma~\ref{lem:x2=q+1} ensures $x = m = 3$.
     Let us now assume $mn \ne 0$.
     Then, Lemma~\ref{lem:2m+3n-square-mod-12} ensures that both $m$ and $n$ are even.
     In particular, both $\mu = m \binover 2$ and $\nu = n \binover 2$ are a positive integers.
     Since 
     $$
     3^n = x^2 - 2^m = x^2 - 2^{2\mu} = (x - 2^\mu)(x + 2^\mu) \,, 
     $$
     there exists $k \in [0, n]$ such that $3^k = x - 2^\mu$ and $3^{n - k} = x + 2^\mu$.
     Besides, inequality $x - 2^\mu < x + 2^\mu$ is clear. 
     It follows $k < n - k$, or equivalently, $3^{n - 2 k} - 1 \in \bNast$.
     Subsequently, straightforward computations yield
     $$
     2^{\mu + 1} = (x + 2^\mu) - (x - 2^\mu) = 3^{n - k} - 3^k = 3^k \left( 3^{n - 2k} - 1 \right) \,,
     $$
     whence $k = 0$ and $2^{\mu + 1} = 3^n - 1$.
     Therefore, 
     %Since the latter equality is equivalent to ${(3^\nu)}^2 = 2^{\mu + 1} + 1$, 
     Lemma~\ref{lem:x2=q+1} ensures $3^\nu \in \{ 2, 3 \}$.
     It follows $\nu = 1$, $n = 2$, $\mu = 2$, $m = 4$, and $x = 5$.
     The desired result is now easily checked by inspection. 
  \end{proof}

\end{document} 

   \begin{proof}
     Assume $x^2 = 2^m + 3^n$.
     Then, Lemma~\ref{lem:indian-modulo} ensures that both $m$ and $n$ are even.
     In particular, there exist $q \in \bN$ such that $n = 2q$.
     $$
     2^m = 3^n - x^2 = 3^{2p} - x^2 = (3^p - x)(3^p + x) 
     $$
     There exist $i$, $j \in \bN$ such that $i + j = m$, $2^i = 3^p - x$, and $2^j = 3^p + x$.
     Since $3^p - x$ is less than $3^q + x$, $i$ is less than $j$, or equivalently, $2^{j - i} - 1$ is an odd integer. 
     $$
     2 \cdot 3^p = (x + 3^p) - (x - 3^p) = 2^j - 2^i = 2^i(2^{j - i} - 1)
     $$
     whence $i = 1$ and $3^p = 2^{j - 1} - 1$
     $(p, j - 1) \in \{ (1, 0) \}$  
\end{proof}


   \begin{lemma} \label{lem:indian-quatre-neuf} 
     For every $x$, $m$, $n \in \bN$, $x^2 = 4^m + 9^n$ is equivalent to $(x, m, n) = (5, 2, 1)$.
   \end{lemma}

   \begin{proof}
     Assume $x^2 = 4^m + 9^n$.
     Then, we have  
     $$
     (x - 2^m)(x + 2^m) = x^2 - 4^m = 9^n = 3^{2n} \,,
     $$
     and thus there exist $i$, $j \in \bN$ such that
     $p + q = 2 n$,
     $x - 2^m = 3^p$, and
     $x + 2^m = 3^q$.
     Since $x - 2^m$  not greater than  $x + 2^m$,
     $p$ is not greater than $q$,
     or equivalently, $3^{q - p}$ is an integer.
     Besides, straightforward computations yield 
     $$
     2^{m + 1} = (x + 2^m) - (x - 2^m) = 3^q - 3^p = 3^p \left(3^{q - p} - 1 \right) \,. 
     $$
     It follows $p = 0$, $q = 2 n$, and $2^{m + 1} = 3^{2n} - 1$,
     whence $9^n - 2^{m + 1} = 1$.
     Now, Lemma~\ref{lem:indian-neuf-deux} ensures $n = 1$ and $m = 2$.    
   \end{proof}
   
\end{document}  
\bibliographystyle{plain}
\bibliography{IndMO92}
\end{document}
   \begin{proof}
     Assume $9^n - 2^m = 1$.
     Then, we have 
     $$
     2^m = 9^n - 1 = (3^n - 1)(3^n + 1) \,, 
     $$
     and thus there exist $p$, $q \in \bN$ such that 
     $3^n - 1 = 2^p$
     and
     $3^n + 1 = 2^q$.
     Since $3^n - 1$ is even, $p$ is positive, or equivalently, $2^{p - 1}$ is an integer;
     since $3^n - 1$ is not greater than $3^n + 1$, $p$ is not greater than $q$, or equivalently, $2^{q - p}$ is an integer.
     Besides, straightforward computations yield 
     $$
     1 =  \frac{3^n + 1}{2} - \frac{3^n - 1}{2} = 2^{q - 1} - 2^{p - 1}  = 2^{p - 1} (2^{q - p} - 1) \, .
     $$
     It follows $p = 1$, whence $n = 1$, and subsequently, $m = 3$.
   \end{proof}

      The table
     $$  
     \begin{array}{|rc|ccccccccc|}
       \hline
       k &            = & 0 & 1 & 2 & 3  & 4  & 5  & 6  & 7  & 8 \\
       \hline
       2^k \bmod 80 & = & 1 & 2 & 4 & 8  & \mathbf{16} & 32 & 64 & 48 & \mathbf{16} \\
       3^k \bmod 80 & = & \mathbf{1} & 3 & 9 & 27 & \mathbf{1}  & 3  & 9  & 27 &  1   \\
       \hline
     \end{array}
     $$
     shows
     that the sequence $\left( 2^k \bmod 80 \right)_{n \in \bN}$ is eventually periodic with 
     $$
     P_2 = \left\{ 2^{k + 4}  \bmod 80 : k \in \bN  \right\} 
     $$
     and
     that the sequence $\left( 3^k \bmod 80 \right)_{n \in \bN}$ is periodic with 
     $$P_3  = \left\{ 3^k \bmod 80 : k \in \bN \right\} \, . 
     $$
     Besides, straightforward computations yield 
     $$
     \left\{ x \bmod 80 : x \in P_2 - P_3 \right\} = D  \,.
     $$
     It follows 
     $$
     D = \left\{ (2^{m + 4} - 3^n) \bmod 80 : (m, n) \in \bN \times \bN \right\} \, .
     $$
     Since $\{ 1, 79 \}$ does not intersect $D$,
     $\abs{2^{m + 4} - 3^n}$ is not congruent to $1$ modulo $80$ for any $m$, $n \in \bN$,
     which implies 
     $$
     S \subseteq \{ 0, 1, 2, 3 \} \times \bN \, .
     $$
     Now, let $(m, n) \in S$.
     \begin{proof}
       Let $x$, $m \in \bN$ be such that $x^2 = 2^m + 1$.
       Since
       $$
       2^m = x^2 - 1 = (x - 1)(x + 1) \,, 
       $$
       both $x - 1$ and $x + 1$ can be written as powers of~$2$.
       Besides, $x - 1$ is less than $x + 1$.
       Therefore, $x - 1$ divides $x + 1$.
       It follows that $x - 1$ divides $(x + 1) - (x - 1) = 2$,
       whence $x - 1 \in \{ 1, 2 \}$, or equivalently, $x \in \{ 2, 3 \}$.
     \end{proof} 
     \begin{proof}
       Let $x$, $p$, $n \in \bN$ be such that  $p$ is prime and $x^2 = p^n + 1$.
       Since
       $$
       p^n = x^2 - 1 = (x - 1)(x + 1) \,, 
       $$
       both $x - 1$ and $x + 1$ can be written as powers of~$p$.
       Besides, $x - 1$ is less than $x + 1$.
       Therefore, $x - 1$ divides $x + 1$.
       It follows that $x - 1$ divides $(x + 1) - (x - 1) = 2$,
       whence $x \in \{ 2, 3 \}$.
       If $x = 2$ then $p^n = 3$, or equivalently, $(p, n) = (3, 1)$.
       If $x = 3$ then $p^n = 8$, or equivalently, $(p, n) = (2, 3)$.
       The desired result is now easily checked by inspection.
    \end{proof} 
